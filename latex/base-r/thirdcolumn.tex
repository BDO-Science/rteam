% !TEX root = base-r.tex

\begin{block}{Programming}
  \vskip-2ex
  \begin{columns}[t]\hfill\hfill\hfill
    \begin{column}{.48\linewidth}
      \begin{subblock}{For Loop}
        \begin{code}
          \begin{Pseudo}
for (variable in sequence) {
    Do something
}
          \end{Pseudo}
        \end{code}
        
        \begin{exampleblock}{Example}
          \begin{Rbg}
for (i in 1:4) {
    j <- i + 10
    print(j)
}
          \end{Rbg}
        \end{exampleblock}
      \end{subblock}
      \begin{subblock}{If Statement}
        \begin{code}
          \begin{Pseudo}
if (condition) {
    Do something
} else {
    Do something
} 
          \end{Pseudo}
        \end{code}
        
        \begin{exampleblock}{Example}
          \begin{Rbg}
if (i > 3) {
    print('Yes')
} else {
    print('No')
} 
          \end{Rbg}
        \end{exampleblock}
      \end{subblock}
    \end{column}\hfill
    \begin{column}{.48\linewidth}
      \begin{subblock}{While Loop}
        
        \begin{code}
          \begin{Pseudo}
while (condition) {
    Do something
}
          \end{Pseudo}
        \end{code}
        
        \begin{exampleblock}{Example}
        
          \begin{Rbg}
while (i < 5) {
    print(i)
    i <- i + 1
}
          \end{Rbg}
        \end{exampleblock}
        
        \end{subblock}
        
        \begin{subblock}{Functions}
          \begin{code}
            \begin{Pseudo}
funct_name <- function(var) {
    Do something
    return(new_variable)
}
            \end{Pseudo}
          \end{code}
          
          \begin{exampleblock}{Example}
          
            \begin{Rbg}
square <- function(x) {
    squared <- x*x
    return(squared)
}
            \end{Rbg}
          \end{exampleblock}
          
        \end{subblock}
      \end{column}\hfill\hfill\hfill
    \end{columns}
    
    \vskip-1.5ex
    \begin{seesubblock}{Reading and Writing Data}{readr}
      \vskip1ex
      \begin{tableau}{| >{\color{black}\centering\small}m{0.27\linewidth} | >{\color{black}\centering\small}m{0.27\linewidth} | >{\color{black}\centering\arraybackslash}m{0.36\linewidth} |}
        \hline
        {\normalfont\textbf{Input}} & {\normalfont\textbf{Output}} & \textbf{Description}\\\hline
        \rowcolor{secondary} \inl{df <- read.table('file.txt')} & \inl{write.table(df, 'file.txt')} & Read and write a delimited text file.\\\hline
        \inl{df <- read.csv('file.csv')} & \inl{write.csv(df, 'file.csv')} & Read and write a comma separated value file. This is a special case of \inl{read.table/write.table}.\\\hline
        \rowcolor{secondary} \inl{load('file.RData')} & \inl{save(df, file = 'file.RData')} & Read and write a n R data file, a file type special for R.\\\hline
      \end{tableau}
      \vskip-1ex
    \end{seesubblock}
    
    \vskip1ex
    \begin{subblock}{}
      \small\hspace{0.25ex}\begin{beamercolorbox}[ht = 3.5ex, dp = 3ex, wd = 0.09\linewidth, center, rounded = true]{block title}
      \textbf{\footnotesize Conditions}
    \end{beamercolorbox}
    \hspace{1ex}
    \footnotesize
    \begin{tableau}{| >{\color{black}}c | >{\color{black}\cellcolor{secondary}}c | >{\color{black}}c | >{\color{black}\centering\cellcolor{secondary}}m{0.075\linewidth} | >{\color{black}}c | >{\color{black}\cellcolor{secondary}\centering\arraybackslash}m{0.1\linewidth} | >{\color{black}}c | >{\color{black}\cellcolor{secondary}}c |}
      \hline
      \inl{a == b} & Are equal & \inl{a > b} & Greater than & \inl{a >= b} & Greater than or equal to & \inl{is.na(a)} & Is missing\\\hline
      \inl{a \!= b} & Not equal & \inl{a < b} & Less than & \inl{a <= b} & Less than or equal to & \inl{is.null(a)} & Is null\\\hline
    \end{tableau}
    \vskip-1ex
  \end{subblock}
\end{block}
